% %%%
% Camp 100 Long Form Document Preamble
% %%%

% basic document setup
\usepackage{geometry}
\geometry{
a4paper,
total={170mm,257mm},
left=20mm,
top=20mm,
}
\usepackage[skip=0.7em]{parskip} % put in nice paragraph breaksq
\setlength\parindent{0pt} % get rid of the indent
\setcounter{tocdepth}{0} % set toc depth to only show chapters
\setcounter{secnumdepth}{3} % use section numbering for everything up too and including subsubsection

% remove hyphenation at end of line
\tolerance=1
\emergencystretch=\maxdimen
\hyphenpenalty=10000
\hbadness=10000

% document global variables, defined for all documents and used to fill blanks etc
% this comes at the top of the doc as we use these values throughout the entire preamble
% we also set here the language settings for the backpage of the document as it's easier to declare this as a global variable
\newcommand{\documentsetup}[6]{%
    \def\documentTitle{#1}
    \def\documentSubtitle{#2}
    \def\publishDate{#3}
    \def\documentID{#4}
    \def\documentLanguage{#5} % ISO 639 compliant 2 letter language code
    \def\documentStatus{#6}

    \ifthenelse{\equal{#5}{en}}{
        \def\backPageCampInfo{Camp 100, a project by Woodcraft Folk will bring together members of all ages from across the world to camp together and live by the Woodcraft Folk values for a week in the summer of 2025. The camp celebrates Woodcraft Folk's 100th birthday and will celebrate it's past century while looking forward to the next 100 years.}
        \def\backPageWcfInfo{Woodcraft Folk is a registered charity in England \& Wales (1148195) and in Scotland (SC039791), and a limited company, registered in England \& Wales (8133727). Registered office: Holyoake House, Hanover Street, Manchester M60 0AS}
        \def\backPageCampSocials{Find Camp 100 on the internet}
        \def\backPageWcfSocials{Find Woodcraft Folk online}
    }%
    {}% false
    \ifthenelse{\equal{#5}{fr}}{
        \usepackage[french]{babel}
        \def\backPageCampInfo{Le camp 100 est un projet de Woodcraft Folk dans lequel des personnes de tout \^age de partout du monde seront rassembler pour camper tous ensemble et vivre les valeurs de Woodcraft Folk pendant 1 semaine pendant l'\'et\'e 2025. Ce camp c\'el\'ebrera les 100 ans de Woodcraft Folk ainsi que le siècle pass\'e tout en se projetant dans les prochaines 100 ann\'ees.}
        \def\backPageWcfInfo{Woodcraft Folk est une organisation caritative enregistr\'ee en Angleterre et aux Pays de Galles (1148195) et en \'ecosse (SC039791), et une entreprise limit\'ee en Angleterre et en \'ecosse (8133727).\\ Bureau enregistr\'e: Holyoake House, Hanover Street, Manchester M60 0AS.}
        \def\backPageCampSocials{Retrouvez Camp 100 sur notre site internet}
        \def\backPageWcfSocials{Retrouvez Woodcraft Folk sur internet}
    }%
    {}% false
    \ifthenelse{\equal{#5}{es}}{
        \usepackage[spanish]{babel}
        \def\backPageCampInfo{Camp 100 es un proyecto de Woodcraft Folk que congregar\'a a miembres de todas las edades y de todo el mundo para acampar juntes y vivir seg\'un los valores de Woodcraft Folk durante una semana en el verano de 2025. El campamento conmemorar\'a el centenario del nacimiento de Woodcraft Folk y celebrar\'a su \'ultimo siglo, con la mirada puesta en los próximos 100 a\~nos.}
        \def\backPageWcfInfo{Woodcraft Folk es una organización sin \'animo de lucro registrada en Inglaterra y Gales (1148195) y en Escocia (SC039791), y una sociedad limitada, registrada en Inglaterra y Gales (8133727). Sede social: Holyoake House, Hanover Street, Manchester M60 0AS.}
        \def\backPageCampSocials{Encuentra Camp 100 en Internet}
        \def\backPageWcfSocials{Encuentra Woodcraft Folk en Internet }
    }%
    {}% false
}



% packages for general use
\usepackage[dvipsnames, table]{xcolor}
\usepackage{graphicx}

\usepackage{tikz}
\usetikzlibrary{calc, shapes.callouts}
\usepackage{adjustbox}
\usepackage{fontawesome}
\usepackage{enumitem}
\usepackage{datetime2}
\usepackage{hyperref}
\usepackage{lastpage}
\usepackage{tcolorbox}
\usepackage{ifthen}
\usepackage{multicol}
\usepackage{multirow}
\usepackage{longtable}
\usepackage{ragged2e}
\usepackage{float}


% configure hyperref to do links & PDF metadata
\hypersetup{
    linktoc=none,
    pdfborderstyle={/S/U/W 1},
    colorlinks=false,
    allbordercolors=blue,
    breaklinks=true,
}
% these ones come in a different block as they must be dealt with at the start of the document
\makeatletter
\AtBeginDocument{
  \hypersetup{
    pdftitle={ \documentTitle },
    pdfauthor={Camp 100},
    pdfsubject={ \documentSubtitle },
    pdfcreator={LaTeX}
  }
}
\makeatother
% redefine \href so it uses text color blue to make links more obvious
\let\oldhref\href
\renewcommand{\href}[2]{\oldhref{#1}{\textcolor{blue}{#2}}}

\newcommand{\internallink}[2]{\hyperref[#1]{\textcolor{blue}{#2}}}


%% COLORS
\definecolor{wcf-accent}{HTML}{6d8f41}
\definecolor{c100-red}{HTML}{F04C58}
\definecolor{c100-green}{HTML}{087F5C}
\definecolor{c100-beige}{HTML}{E9E1CA}
\definecolor{c100-orange}{HTML}{EC7E2C}
\definecolor{c100-grey}{HTML}{545454}

% setup setup fonts
% we manually specify to use the `_Light_Oblique' varient of Quicksand for italics as the font doesn't include an italic face by default.
\usepackage{fontspec}
\setmainfont[Ligatures=TeX, ItalicFont=*_Light_Oblique]{Quicksand}
\setsansfont[Ligatures=TeX, ItalicFont=*_Light_Oblique]{Quicksand}

% tables
\renewcommand{\arraystretch}{1.6} % make cells vertically bigger

\newcommand{\tablehead}[1]{\centering\arraybackslash \cellcolor{c100-grey}\leavevmode\color{white}\textbf{#1}} % we try this as it actually allows automatic linebreaks in the headings
% \newcommand{\tablehead}[1]{\multicolumn{1}{c}{\cellcolor{c100-red}\textcolor{white}{\textbf{#1}}}}



% Document Title Page
% adapted from: https://www.reddit.com/r/LaTeX/comments/faij1n/my_first_cover_page_done_in_latex_is_it/
\newcommand{\makedocumenttitlepage}{\begin{titlepage}
    \begin{tikzpicture}[overlay,remember picture]

        % \fill[black!2] (current page.south west) rectangle (current page.north east);
        
        % line 01
        \begin{scope}[transform canvas ={rotate around ={45:($(current page.north west)+(-.5,-6)$)}}]
        \shade[rounded corners=18pt, left color=c100-orange, right color=c100-orange] ($(current page.north west)+(-.5,-6)$) rectangle ++(9,1.5);
        \end{scope}
        
        % line 02
        \begin{scope}[transform canvas ={rotate around ={45:($(current page.north west)+(.5,-10)$)}}]
        \shade[rounded corners=18pt, left color=c100-green ,right color=c100-green] ($(current page.north west)+(0.5,-10)$) rectangle ++(15,1.5);
        \end{scope}
        
        % line 03
        \begin{scope}[transform canvas ={rotate around ={45:($(current page.north)+(-1.5,-3)$)}}]
        \shade[rounded corners=12pt, left color=c100-grey, right color=c100-grey] ($(current page.north)+(-1.5,-3)$) rectangle ++(9,0.8);
        \end{scope}
        
        % line 04
        \begin{scope}[transform canvas ={rotate around ={45:($(current page.north)+(-3,-8)$)}}]
        \shade[rounded corners=28pt, left color=c100-red, right color=c100-red] ($(current page.north)+(-3,-8)$) rectangle ++(15,1.8);
        \end{scope}
        
        % line 05
        \begin{scope}[transform canvas ={rotate around ={45:($(current page.north west)+(4,-15.5)$)}}]
        \shade[rounded corners=25pt, left color=c100-green, right color=c100-green] ($(current page.north west)+(4,-15.5)$) rectangle ++(30,1.8);
        \end{scope}
        
        % line 06
        \begin{scope}[transform canvas ={rotate around ={45:($(current page.north west)+(13,-10)$)}},]
        \shade[rounded corners=22pt, left color=c100-orange,right color=c100-orange] ($(current page.north west)+(13,-10)$) rectangle ++(15,1.5);
        \end{scope}
        
        % line 07
        \begin{scope}[transform canvas ={rotate around ={45:($(current page.north west)+(19,-5.65)$)}},]
        \shade[rounded corners=12pt, left color=c100-grey,right color=c100-grey] ($(current page.north west)+(19,-5.65)$) rectangle ++(15,0.8);
        \end{scope}
        
        % line 08
        \begin{scope}[transform canvas ={rotate around ={45:($(current page.north west)+(18,-8)$)}},]
        \shade[rounded corners=8pt, left color=c100-green,right color=c100-green] ($(current page.north west)+(18,-8)$) rectangle ++(15,0.6);
        \end{scope}
        
        % line 09
        \begin{scope}[transform canvas ={rotate around ={45:($(current page.north west)+(20,-9)$)}}]
        \shade[rounded corners=20pt, left color=c100-red, right color=c100-red] ($(current page.north west)+(20,-9)$) rectangle ++(14,1.2);
        \end{scope}
        
        \node[inner sep=0pt] () at ($(current page.center)+(5,-11)$){\includegraphics[width=0.45\textwidth]{../global-logo-c100-wide.png}};
        \node[inner sep=0pt] () at ($(current page.center)+(-5,-11)$){\includegraphics[width=0.45\textwidth]{../global-logo-wcf-wide.png}};
        
        \node[text width=\textwidth,align=center] at ($(current page.center)+(0,-3.5)$) {{\Huge \documentTitle}};
        
        \node[text width=\textwidth,align=center] at ($(current page.center)+(0,-5.5)$) {{\Large{\textit{\documentSubtitle}}} \\[2em] \large \publishDate};
        
        \end{tikzpicture}
\end{titlepage}
\scalefont{\standardfontscale}
}

% Document Back Page
\newcommand{\makedocumentbackpage}{\newpage
    \scalefont{\backpagefontscale}
  \thispagestyle{empty}
  
  \begin{tikzpicture}[overlay,remember picture]

      % \fill[black!2] (current page.south west) rectangle (current page.north east);
      
      % line 01
      \begin{scope}[transform canvas ={rotate around ={45:($(current page.north west)+(-.5,-6)$)}}]
      \shade[rounded corners=18pt, left color=c100-orange, right color=c100-orange] ($(current page.north west)+(-.5,-6)$) rectangle ++(9,1.5);
      \end{scope}
      
      % line 02
      \begin{scope}[transform canvas ={rotate around ={45:($(current page.north west)+(.5,-10)$)}}]
      \shade[rounded corners=18pt, left color=c100-green ,right color=c100-green] ($(current page.north west)+(0.5,-10)$) rectangle ++(15,1.5);
      \end{scope}
      
      % line 03
      \begin{scope}[transform canvas ={rotate around ={45:($(current page.north)+(-1.5,-3)$)}}]
      \shade[rounded corners=12pt, left color=c100-grey, right color=c100-grey] ($(current page.north)+(-1.5,-3)$) rectangle ++(9,0.8);
      \end{scope}
      
      % line 04
      \begin{scope}[transform canvas ={rotate around ={45:($(current page.north)+(-3,-8)$)}}]
      \shade[rounded corners=28pt, left color=c100-red, right color=c100-red] ($(current page.north)+(-3,-8)$) rectangle ++(15,1.8);
      \end{scope}
      
      % line 05
      \begin{scope}[transform canvas ={rotate around ={45:($(current page.north west)+(4,-15.5)$)}}]
      \shade[rounded corners=25pt, left color=c100-green, right color=c100-green] ($(current page.north west)+(4,-15.5)$) rectangle ++(30,1.8);
      \end{scope}
      
      % line 06
      \begin{scope}[transform canvas ={rotate around ={45:($(current page.north west)+(13,-10)$)}},]
      \shade[rounded corners=22pt, left color=c100-orange,right color=c100-orange] ($(current page.north west)+(13,-10)$) rectangle ++(15,1.5);
      \end{scope}
      
      % line 07
      \begin{scope}[transform canvas ={rotate around ={45:($(current page.north west)+(19,-5.65)$)}},]
      \shade[rounded corners=12pt, left color=c100-grey,right color=c100-grey] ($(current page.north west)+(19,-5.65)$) rectangle ++(15,0.8);
      \end{scope}
      
      % line 08
      \begin{scope}[transform canvas ={rotate around ={45:($(current page.north west)+(18,-8)$)}},]
      \shade[rounded corners=8pt, left color=c100-green,right color=c100-green] ($(current page.north west)+(18,-8)$) rectangle ++(15,0.6);
      \end{scope}
      
      % line 09
      \begin{scope}[transform canvas ={rotate around ={45:($(current page.north west)+(20,-9)$)}}]
      \shade[rounded corners=20pt, left color=c100-red, right color=c100-red] ($(current page.north west)+(20,-9)$) rectangle ++(14,1.2);
      \end{scope}

    \end{tikzpicture}
      
    % textual boxes at the bottom of the page
    \begin{tikzpicture}[overlay,remember picture]
      % top left box, textual information about C100
      \node[rectangle, anchor=north west,
              rounded corners=18pt,inner sep=11pt,
              fill=c100-green,
              minimum height=11em] () at
              ($(current page.center)+(-9, -3)$)
              {\parbox[t]{0.55\textwidth}{\color{white} \backPageCampInfo}};

      % top right box, social media contact for C100
      \node[rectangle, anchor=north west,
              rounded corners=18pt,inner sep=11pt,
              fill=c100-red,
              minimum height=11em] () at
              ($(current page.center)+(2,-3)$)
              {\parbox[t]{0.35\textwidth}{\color{white} \backPageCampSocials\\[-0.5em] \begin{itemize}[leftmargin=1.5em]
                \setlength\itemsep{0.7em}
                \item[\faInstagram] camp100wcf
                \item[\faFacebook] camp100wcf
                \item[\faLink] camp100.org.uk
              \end{itemize}}};
      
      % bottom left box, WCF charity disclaimer
      \node[rectangle, anchor=north west,
              rounded corners=18pt,inner sep=11pt,
              fill=c100-red,
              minimum height=11em] () at
              ($(current page.center)+(-9,-8)$)
              {\parbox[t]{0.4\textwidth}{\color{white} \backPageWcfInfo}};
      % bottom middle, WCF social media
      \node[rectangle, anchor=north west,
              rounded corners=18pt,inner sep=11pt,
              fill=c100-orange,
              minimum height=11em] () at
              ($(current page.center)+(-0.95,-8)$)
              {\parbox[t]{0.25\textwidth}{\color{white} \backPageWcfSocials\\[-0.5em] \begin{itemize}[leftmargin=1.5em]
                \setlength\itemsep{0.7em}
                \item[\faInstagram] woodcraftfolk
                \item[\faFacebook] woodcraftfolk
                \item[\faLink] woodcraft.org.uk
              \end{itemize}}};
      % bottom right box, WCF logo
      \node[rectangle, anchor=north west,
              rounded corners=18pt,inner sep=11pt,
              fill=c100-green,
              minimum height=11em] () at
              ($(current page.center)+(4.5,-8)$)
              {\parbox[t]{0.2\textwidth}{\color{white} \includegraphics[width=0.2\textwidth]{../global-logo-wcf-white.png}}};
      % very bottom line, document information
      \node[rectangle, anchor=north west,
              rounded corners=12pt,inner sep=11pt,
              fill=c100-grey] () at
              ($(current page.center)+(-9,-13)$)
              {\parbox[t]{0.99\textwidth}{\color{white} \footnotesize{\texttt{\documentID}\ (\texttt{\jobname .tex}) compiled at \texttt{\DTMnow}\ with version \texttt{\documentStatus} in language \texttt{\documentLanguage}}}};
    \end{tikzpicture}
  

}

% section titles
% chapter heading adapted from: https://texblog.net/latex-archive/uncategorized/fancy-chapter-tikz/
\usepackage[explicit]{titlesec}

% we only want to deal with chapters if document class is a report; as it'll break in articles otherwise
\makeatletter
\@ifclassloaded{report}{%
    \titlespacing*{\chapter}{0pt}{30pt}{60pt}

    \titleformat{\chapter}
    {\gdef\chapterlabel{}
    \normalfont\sffamily\Huge\bfseries}
    {\gdef\chapterlabel{\thechapter\ | }}{0pt}
    {\begin{tikzpicture}[remember picture,overlay]
        % line 01
        \begin{scope}[transform canvas ={rotate around ={45:($(current page.north west)+(13,-10)$)}},]
        \shade[rounded corners=12pt, left color=c100-grey,right color=c100-grey] ($(current page.north west)+(21,-7)$) rectangle ++(15,0.8);
        \end{scope}
        % line 02
        \begin{scope}[transform canvas ={rotate around ={45:($(current page.north west)+(13,-10)$)}},]
        \shade[rounded corners=20pt, left color=c100-red,right color=c100-red] ($(current page.north west)+(17,-9)$) rectangle ++(15,1.3);
        \end{scope},,inner sep=11pt,
                fill=c100-green] 
                {\parbox[t]{0.6\textwidth}{\color{white}\chapterlabel#1}};
    \end{tikzpicture}
    }
    % redefine \chapter so it uses pagestyle=fancy
    \makeatletter
    \renewcommand\chapter{\if@openright\cleardoublepage\else\clearpage\fi
    \thispagestyle{fancy}%
    \global\@topnum\z@
    \@afterindentfalse
    \secdef\@chapter\@schapter}
    \makeatother
}
\makeatother



\titleformat{\section}
    {\normalfont\LARGE\sffamily\bfseries\color{black}}
    {\tcbox[colback=c100-green, colframe=c100-green, coltext=white, on line, boxsep=0pt, left=4pt, right=4pt, top=4pt, bottom=4pt]{\scalefont{\standardfontscale}\thesection}}{0.2em}{\scalefont{\standardfontscale}#1}
\titleformat{\subsection}
    {\normalfont\Large\sffamily\bfseries\color{black}}
    {\tcbox[colback=c100-green, colframe=c100-green, coltext=white, on line, boxsep=0pt, left=4pt, right=4pt, top=4pt, bottom=4pt]{\scalefont{\standardfontscale}\thesubsection}}{0.2em}{\scalefont{\standardfontscale}#1}
\titleformat{\subsubsection}
    {\normalfont\large\sffamily\bfseries\color{black}}
    {\tcbox[colback=c100-green, colframe=c100-green, coltext=white, on line, boxsep=0pt, left=4pt, right=4pt, top=4pt, bottom=4pt]{\scalefont{\standardfontscale}\thesubsubsection}}{0.2em}{\scalefont{\standardfontscale}#1}

% bigger fonts
% we're being lazy here - only looking at the main content & section headers for these - could come back to be better at a later date.
\usepackage{scalefnt}
% set some defaults for non scaled up docs
\def\backpagefontscale{1}
\def\standardfontscale{1}
% enlarge
\newcommand{\uselargefont}{%
\def\backpagefontscale{0.5}
\def\standardfontscale{2}
}



% headers and footers
\usepackage{fancyhdr}


\pagestyle{fancy}
\fancyhf{}
\fancyfoot[L]{\documentTitle}
\fancyfoot[C]{\textbf{\thepage}\ of \pageref*{LastPage}}
\fancyfoot[R]{\publishDate}
\renewcommand{\footrulewidth}{0.4pt}
\renewcommand{\headrulewidth}{0.0pt}
\addtolength{\topmargin}{-1.59999pt}
\setlength{\headheight}{13.59999pt}

% now we do headers and footers for non-release version watermarks
\AtBeginDocument{
    \usepackage{draftwatermark}
    \DraftwatermarkOptions{
        stamp=false,
        text={\documentStatus},
        color=red!30
    }
    \ifthenelse{\equal{\documentStatus}{proof}}{
        \DraftwatermarkOptions{stamp=true}
    }%
    {}% if not set to 1
    \ifthenelse{\equal{\documentStatus}{draft}}{
        \DraftwatermarkOptions{stamp=true}
    }%
    {}% if not set to 1

    
}

% callouts

\usepackage[framemethod=TikZ]{mdframed}
\mdfsetup{skipabove=\topskip,
skipbelow=\topskip,
leftmargin=0cm,
rightmargin=0cm,
linewidth=3pt
}

% Red callout
\mdfdefinestyle{callout-red}{%
linecolor=c100-red!70,
backgroundcolor=c100-red!10,
topline=false,
bottomline=false,
rightline=false,
frametitlebackgroundcolor=c100-red!30,
}
\newenvironment{callout-red}[1]
    {\begin{mdframed}[style=callout-red,frametitle={#1}]
        }
        {
    \end{mdframed}
    }

% Orange callout
\mdfdefinestyle{callout-orange}{%
linecolor=c100-orange!70,
backgroundcolor=c100-orange!10,
topline=false,
bottomline=false,
rightline=false,
frametitlebackgroundcolor=c100-orange!30,
}
\newenvironment{callout-orange}[1]
    {\begin{mdframed}[style=callout-orange,frametitle={#1}]
        }
        {
    \end{mdframed}
    }

% Green callout
\mdfdefinestyle{callout-green}{%
linecolor=c100-green!70,
backgroundcolor=c100-green!10,
topline=false,
bottomline=false,
rightline=false,
frametitlebackgroundcolor=c100-green!30,
}
\newenvironment{callout-green}[1]
    {\begin{mdframed}[style=callout-green,frametitle={#1}]
        }
        {
    \end{mdframed}
    }

% bold callouts - to be used sparingly

\mdfdefinestyle{bold-callout-generic}{
linewidth=0.0em,
fontcolor=white,
roundcorner=0.25em,
font=\bfseries
}
\newenvironment{callout-bold-green}
    {\begin{mdframed}[style=bold-callout-generic, backgroundcolor=c100-green]
        }
        {
    \end{mdframed}
    }
\newenvironment{callout-bold-orange}
    {\begin{mdframed}[style=bold-callout-generic, backgroundcolor=c100-orange]
        }
        {
    \end{mdframed}
    }
\newenvironment{callout-bold-red}
    {\begin{mdframed}[style=bold-callout-generic, backgroundcolor=c100-red]
        }
        {
    \end{mdframed}
    }


% TIKZ
\tikzset{
    c100-speech-callout/.style= {color = white,
                        draw,
                        shape=rectangle callout,
                        fill = c100-green,
                        rounded corners=0.25em,
                        line width = 0.1em,
                        inner xsep = 0.5em,
                        inner ysep = 0.5em,
                        callout relative pointer={(1.25cm,-1cm)},
                        callout pointer width=2cm,
                        font=\bfseries},
    c100-sc-left/.style= {callout relative pointer={(-1.25cm,-1cm)}}
}

% QR Codes
\usepackage{qrcode}
\usepackage{wrapfig}

\newcommand{\insertqrcode}[2]{%
\begin{wrapfigure}{r}{0.35\textwidth}
  \centering
  \begin{mdframed}[linecolor=c100-green, linewidth=0.2em, roundcorner=0.25em, skipabove=0em, skipbelow=0em]
    \begin{center}
      \qrcode[height=0.9\textwidth, nolinks, level=M, version=4]{#1}
      
      {\small \vspace{0.5em} \href{#1}{#2}}
    \end{center}
    
  \end{mdframed}
\end{wrapfigure}
}